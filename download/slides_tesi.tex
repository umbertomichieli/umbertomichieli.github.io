% Compile it with PDFLatTeX

\documentclass[]{beamer}
\usetheme{CambridgeUS} % theme
\usepackage[italian]{babel}
\usepackage{latexsym}
\usepackage{graphicx}
\usepackage{float} %gestisce l'inserimento delle immagini
\usepackage{setspace} %consente di modificare l'interlinea
\usepackage{subfigure}
\usepackage{longtable}
\usepackage{amssymb}
\usepackage{wrapfig}
\usepackage{url}
\usepackage{array}
\usepackage[T1]{fontenc} 
\usepackage[utf8]{inputenc}
\usepackage{textcomp}
\newcommand{\argmax}{\operatornamewithlimits{argmax}}
\definecolor{blu}{rgb}{0.216,0.216,0.6745}

\setbeamercovered{transparent} % \pause text visibility

%------------------------------ TITLE PAGE DECLARATION
\title[Correlation analysis of EEG and EMG]{\textsc{Analisi della correlazione tra segnali EEG ed EMG}}
\author[Umberto Michieli]{Laureando: Umberto Michieli\\Relatore: Leonardo Badia\\Correlatrice: Giulia Cisotto}
\date[18/07/2016]{18/07/2016\\Anno accademico 2015/2016}
\institute[]{Corso di Laurea in Ingegneria dell'Informazione\\ Dipartimento di Ingegneria dell'Informazione}
%\logo{\includegraphics[width=1cm]{images/logo_unipd.pdf}}
\titlegraphic{\begin{flushleft}
\includegraphics[scale=0.2]{images/dei_logo.pdf} \quad \quad \quad \quad \quad \quad \quad \quad \quad \quad \quad \quad \quad \quad \quad \quad \quad \quad \quad \quad \quad \quad  \  \includegraphics[scale=0.07]{images/logo_unipd.pdf}\end{flushleft}
%\begin{flushright}
%\includegraphics[scale=1]{images/logo_unipd.pdf}\end{flushright}
}

%------------------------------ /TITLE PAGE DECLARATION

\begin{document}
\transduration{1}


\frame{\titlepage} %------------------------------ TITLE PAGE


\section*{Sommario}
\begin{frame} %------------------------------ OUTLINE
\transwipe[direction=0] % book-like transition
\frametitle{Sommario}
\tableofcontents
\end{frame}


\section{Obiettivo}

\begin{frame}
\transwipe[direction=0]
%\frametitle{}

EEG=Elettroencefalogramma\\
EMG=Elettromiogramma\\
APB=muscolo abduttore breve del pollice
\begin{block}{}
Gli eventi patologici \textit{burst} nell'EMG sono provocati da uno stimolo cerebrale o locale? \\ \textcolor{blu}{$\Longrightarrow$} Analisi di correlazione e coerenza tra EEG ed EMG

\end{block}

EMG APB sano, attività di fondo \quad \quad EMG APB con eventi burst

\end{frame}


\section{Conclusioni e sviluppi futuri}
\begin{frame}
\transwipe[direction=0]
\frametitle{}
\begin{block}{Conclusioni}
\begin{itemize}
\item massimo di correlazione con EEG in anticipo in media di 10 ms su EMG, confermando la fisiologia umana
\item coerenza caso sano: picchi di coerenza a 20 Hz e a frequenze inferiori
\item coerenza caso patologico: picchi di coerenza significativa a 20 e attorno a 35-40 Hz
\end{itemize}
\end{block}

\begin{block}{Sviluppi futuri}
\begin{itemize}
\item allargare lo studio per rendere le stime più robuste
\item verificare se gli eventi \textit{burst} sono volontari oppure no
\end{itemize}
\end{block}
\end{frame}

\end{document}